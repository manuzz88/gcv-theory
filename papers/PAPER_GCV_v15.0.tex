\documentclass[12pt,a4paper]{article}
\usepackage[utf8]{inputenc}
\usepackage[T1]{fontenc}
\usepackage{amsmath,amssymb}
\usepackage{booktabs}
\usepackage{hyperref}
\usepackage{geometry}
\usepackage{natbib}
\geometry{margin=2.5cm}
\hypersetup{colorlinks=true,linkcolor=blue,citecolor=blue,urlcolor=blue}

\title{\textbf{Vacuum Coherence Gravity v15.0}\\
\large A Unified Theory of Dark Matter and Dark Energy\\
from Quantum Vacuum Organization}
\author{Manuel Lazzaro\\
\small Independent Researcher, Italy\\
\small \href{mailto:manuel.lazzaro@me.com}{manuel.lazzaro@me.com}\\
\small Code: \href{https://github.com/manuzz88/gcv-theory}{github.com/manuzz88/gcv-theory}}
\date{February 2026 --- Version 15.1 (Preprint, not peer reviewed)\\
\small DOI: \href{https://doi.org/10.5281/zenodo.17505641}{10.5281/zenodo.17505641}}

\begin{document}
\maketitle

\begin{abstract}
We present Vacuum Coherence Gravity v15.0 (GCV), a modified gravity framework in which the quantum vacuum responds dynamically to local matter density, unifying dark matter and dark energy phenomenology from a single scalar field. The central function $\Gamma(\rho) = \tanh(\rho/\rho_t)$ --- derived as the exact domain-wall solution of a k-essence Lagrangian --- transitions between gravitational enhancement in dense regions (dark matter effect) and vacuum-driven expansion in voids (dark energy effect).

\textbf{Galactic scale}: Applied to the full SPARC catalog (175 galaxies), GCV reproduces rotation curves with 0.06\% mean deviation using the MOND acceleration scale $a_0 = 1.2\times10^{-10}$ m/s$^2$, with no free parameters per galaxy.

\textbf{Cosmological scale}: We modified the C source code of the CLASS Boltzmann solver (ESA/Planck standard tool) to implement GCV as modified gravity with $\mu(a) = 1 + \mu_0\,\Omega_\mathrm{DE}(a)$. With one new parameter $\mu_0 = 0.15$, we find $\Delta\chi^2 = -17.70$ relative to $\Lambda$CDM on a combined dataset (CMB TT, $f\sigma_8$, $S_8$), reducing the $S_8$ tension from $3.9\sigma$ to $2.6\sigma$ while preserving CMB acoustic peaks ($<0.5\%$) and the BAO sound horizon ($r_s = 147.11$ Mpc, identical to $\Lambda$CDM).

\textbf{Important caveat}: The $\chi^2$ uses a simplified likelihood, not the full Planck plik analysis. $\mu_0 = 0.15$ is fitted, not derived from first principles. These results are preliminary.

\textbf{New in v15.1}: We apply GCV to CDG-2 \citep{Li2025}, a candidate dark galaxy in the Perseus cluster reported to be $99.94\%$ dark matter. GCV predicts the extreme apparent dark matter fraction naturally (deep-MOND regime, $g_N/a_0 \sim 0.006$) and makes a testable prediction for the velocity dispersion: $\sigma \approx 21$ km/s, measurable with follow-up spectroscopy.

GCV makes four falsifiable predictions testable by DESI/Euclid by 2028. All code is open source.

\textbf{Keywords}: modified gravity, dark matter alternatives, dark energy, quantum vacuum, galaxy rotation curves, CMB, S8 tension, CLASS Boltzmann solver, dark galaxies
\end{abstract}

\tableofcontents
\newpage

\section{Introduction}

\subsection{The Dark Matter and Dark Energy Problems}

The standard $\Lambda$CDM model postulates $\sim27\%$ cold dark matter and $\sim68\%$ cosmological constant \citep{Planck2020}. Despite remarkable success, it faces persistent challenges: no direct dark matter detection after 50 years \citep{LUX2017}; the $S_8$ tension ($3$--$4\sigma$ discrepancy between CMB and weak lensing $\sigma_8$ \citealt{DES2022,KiDS2021}); the $H_0$ tension ($\sim5\sigma$, \citealt{Riess2022}); and the tight Radial Acceleration Relation (RAR) in galaxies \citep{McGaugh2016}, which is natural in MOND-like theories.

\subsection{Core Idea}

We propose that the quantum vacuum responds dynamically to local matter density via:
\begin{equation}
    \Gamma(\rho) = \tanh\!\left(\frac{\rho}{\rho_t}\right), \quad \rho_t = \Omega_\Lambda\,\rho_\mathrm{crit}
    \label{eq:gamma}
\end{equation}
\begin{itemize}
    \item $\rho \gg \rho_t$ (galaxies): $\Gamma \to 1$, gravity enhanced $\to$ \textbf{dark matter effect}
    \item $\rho \ll \rho_t$ (voids): $\Gamma \to 0$, vacuum energy dominates $\to$ \textbf{dark energy effect}
\end{itemize}
This is not an ad-hoc ansatz: it is the exact analytical domain-wall solution of a k-essence Lagrangian (Section~\ref{sec:theory}).

\section{Theoretical Framework}
\label{sec:theory}

\subsection{Lagrangian Derivation}

The GCV scalar field $\phi$ obeys a k-essence Lagrangian with symmetry-breaking potential:
\begin{equation}
    \mathcal{L} = -\tfrac{1}{2}(\partial_\mu\phi)^2 - \tfrac{\lambda}{4}(\phi^2 - v^2)^2
\end{equation}
The static domain-wall solution is $\phi(x) = v\tanh(x/\delta)$, with $\delta = \sqrt{2/\lambda}/v$. Identifying the field amplitude with $\rho/\rho_t$ yields Eq.~(\ref{eq:gamma}) exactly.

\subsection{Galactic Scale: Modified Poisson Equation}

GCV modifies the Poisson equation:
\begin{equation}
    \nabla\cdot[(1+\chi_v)\nabla\Phi] = 4\pi G\rho_b
\end{equation}
In the deep-MOND regime ($g \ll a_0$): $\chi_v(g) = \sqrt{a_0/g} - 1$, reproducing the MOND interpolating function and the Baryonic Tully-Fisher relation $v_\infty^4 = GM_b a_0$.

\subsection{Cosmological Scale: Modified Gravity}

At cosmological scales, GCV modifies the effective gravitational coupling for perturbations:
\begin{equation}
    \mu(a) = 1 + \mu_0\,\Omega_\mathrm{DE}(a)
    \label{eq:mu}
\end{equation}
where $\Omega_\mathrm{DE}(a) = \Omega_\Lambda/[\Omega_\Lambda + \Omega_m a^{-3}]$. This modifies the Poisson equation in Newtonian gauge:
\begin{equation}
    k^2\Phi = -4\pi G a^2\,\mu(a)\sum_i\rho_i\delta_i
\end{equation}
\textbf{Key property}: $\mu(a)\to 1$ for $z\gtrsim 10$, so CMB acoustic physics and BBN are unaffected. The modification is active only in the late universe.

\subsection{Note on the Galactic--Cosmological Connection}

The two formulations share the same physical origin (vacuum response to density) but are not yet formally connected. Deriving $\mu(a)$ from $\Gamma(\rho) = \tanh$ through the perturbation equations is left for future work.

\section{Galactic Scale Tests: SPARC 175 Galaxies}
\label{sec:galactic}

\subsection{Method}

We apply the Baryonic Tully-Fisher relation to all 175 SPARC galaxies \citep{Lelli2016}:
\begin{equation}
    v_\infty = (GM_b\,a_0)^{1/4}, \quad a_0 = 1.2\times10^{-10}\text{ m/s}^2
\end{equation}
No free parameters per galaxy. We compute MAPE between predicted and observed $v_\infty$.

\subsection{Results}

\begin{table}[h]
\centering
\caption{SPARC results (175 galaxies)}
\begin{tabular}{lcc}
\toprule
Metric & Value & Note \\
\midrule
Mean deviation & 0.06\% & All 175 galaxies \\
MAPE & 10.7\% & Asymptotic velocity \\
Free parameters per galaxy & 0 & Fixed $a_0$ only \\
\bottomrule
\end{tabular}
\end{table}

The 0.06\% mean deviation reflects the tight BTFR fit. The 10.7\% MAPE is consistent with observational scatter and inclination uncertainties in SPARC.

\subsection{ISW Signal}

GCV predicts enhanced ISW from supervoids: $\Delta T_\mathrm{ISW}^\mathrm{GCV} \approx 1.76\times\Delta T_\mathrm{ISW}^{\Lambda\mathrm{CDM}}$. Observed: $-11.3\,\mu$K \citep{Kovacs2022}; $\Lambda$CDM: $-9\,\mu$K; GCV: $-16\,\mu$K. The observation lies between the two predictions, consistent with the GCV direction but not conclusive.

\section{Cosmological Scale: Modified CLASS Boltzmann Solver}
\label{sec:cosmological}

\subsection{CLASS Modification}

We modified CLASS v3.3.4 \citep{Blas2011} C source code in three files:
\begin{itemize}
    \item \texttt{include/background.h}: Added \texttt{gcv\_mu\_0} to the \texttt{background} struct
    \item \texttt{source/input.c}: Parameter reading with default \texttt{gcv\_mu\_0 = 0}
    \item \texttt{source/perturbations.c}: Applied $\mu(a)$ to Poisson and anisotropic stress in both Newtonian and synchronous gauges
\end{itemize}
The background evolution (Friedmann equations, $H(z)$, $r_s$) is identical to $\Lambda$CDM.

\subsection{Results}

\begin{table}[h]
\centering
\caption{Cosmological observables: $\Lambda$CDM vs GCV ($\mu_0 = 0.15$)}
\begin{tabular}{lccc}
\toprule
Observable & $\Lambda$CDM & GCV ($\mu_0=0.15$) & Observed \\
\midrule
$\sigma_8$ & 0.8229 & \textbf{0.8016} & --- \\
$S_8$ & 0.8416 & \textbf{0.8198} & $0.776\pm0.017$ (DES) \\
$r_s$ (Mpc) & 147.11 & \textbf{147.11} & $147.09\pm0.26$ (Planck) \\
$S_8$ tension vs DES & $3.9\sigma$ & \textbf{$2.6\sigma$} & --- \\
$\Delta\chi^2$ & 0 & \textbf{$-17.70$} & negative = better \\
CMB peaks change & --- & $<0.5\%$ & --- \\
\bottomrule
\end{tabular}
\end{table}

\subsection{Statistical Interpretation}

By the likelihood ratio test with 1 additional parameter: $\Delta\chi^2 < -10$ constitutes decisive evidence on the Jeffreys scale. GCV achieves $\Delta\chi^2 = -17.70$.

\textbf{Important caveat}: This uses a simplified likelihood (not full Planck plik+Commander+lowl+lensing). A full MCMC with all cosmological parameters free is required for a definitive conclusion. This result is a strong preliminary indication.

\subsection{What is Preserved}

\begin{itemize}
    \item CMB acoustic peaks: $<0.5\%$ change in $C_\ell^{TT}$ for $\ell < 2000$
    \item BAO sound horizon: $r_s = 147.11$ Mpc, identical to $\Lambda$CDM
    \item Big Bang nucleosynthesis: $\mu(a)\to 1$ at $z>10$
    \item Gravitational wave speed: tensor modes unaffected; consistent with GW170817 ($|c_\mathrm{gw}-c|/c < 10^{-15}$)
\end{itemize}

\section{Extreme Test Case: CDG-2 Dark Galaxy}
\label{sec:cdg2}

CDG-2 (Candidate Dark Galaxy-2) is a recently discovered galaxy in the Perseus cluster \citep{Li2025}, consisting of four globular clusters (GCs) embedded in extremely faint diffuse emission. It represents one of the most dark-matter-dominated galaxies ever found and provides a stringent test for GCV.

\subsection{Observed Properties}

From \citet{Li2025} (arXiv:2506.15644):
\begin{itemize}
    \item Distance: 75 Mpc (Perseus cluster)
    \item Total luminosity: $L_{V,\mathrm{gal}} = 6.2 \pm 3.0 \times 10^6\,L_\odot$
    \item Stellar mass: $M_* \approx 1.2 \times 10^7\,M_\odot$ ($M/L_V \sim 2$)
    \item GC span diameter: $\sim 1.2$ kpc (4 GCs)
    \item Mean surface brightness: $\langle\mu\rangle_V \sim 27.5$ mag arcsec$^{-2}$
    \item DM halo mass (GC-to-halo relation): $2$--$5.7 \times 10^{10}\,M_\odot$
    \item Inferred DM fraction: 99.94\%--99.98\%
\end{itemize}

\textbf{Critical note}: The dark matter fraction is \textit{not} measured dynamically. It is inferred from the empirical GC-to-halo mass relation \citep{Harris2017,Burkert2020}. No velocity dispersion has been measured yet.

\subsection{GCV Analysis}

At the effective radius ($R_\mathrm{eff} \sim 1.5$ kpc), the Newtonian acceleration from baryonic mass alone is:
\begin{equation}
    \frac{g_N}{a_0} = \frac{G M_*}{a_0 R_\mathrm{eff}^2} \approx 0.006
\end{equation}
This places CDG-2 deep in the MOND/GCV regime ($g_N \ll a_0$), where vacuum susceptibility provides an enhancement factor of $\sim 13\times$.

The apparent dynamical mass at $R_\mathrm{eff}$ is:
\begin{equation}
    M_\mathrm{dyn}^\mathrm{GCV}(R_\mathrm{eff}) \approx 1.6 \times 10^8\,M_\odot
\end{equation}
corresponding to an apparent dark matter fraction of $\sim 92\%$ at 1.5 kpc, rising to $> 96\%$ at 3 kpc and increasing further at larger radii.

\subsection{Testable Prediction}

In the deep-MOND regime, the velocity dispersion of a pressure-supported system depends only on baryonic mass and $a_0$:
\begin{equation}
    \sigma_\mathrm{GCV} \sim (G\,M_*\,a_0)^{1/4} \approx 21\,\mathrm{km/s}
    \label{eq:sigma_cdg2}
\end{equation}
For comparison, the Newtonian prediction (no dark matter) gives $\sigma_\mathrm{Newton} \approx 6$ km/s at $R_\mathrm{eff}$.

\begin{table}[h]
\centering
\caption{GCV prediction for CDG-2 velocity dispersion}
\begin{tabular}{lcc}
\toprule
Baryonic mass assumption & $\sigma_\mathrm{GCV}$ (km/s) & Note \\
\midrule
$M_b = 6 \times 10^6\,M_\odot$ (low) & 17.6 & Lower $L_V$ bound \\
$M_b = 1.2 \times 10^7\,M_\odot$ (best) & \textbf{20.9} & Best estimate \\
$M_b = 1.8 \times 10^7\,M_\odot$ (high) & 23.1 & Upper $L_V$ bound \\
$M_b = 2.4 \times 10^7\,M_\odot$ (+gas) & 24.9 & With 50\% gas \\
\bottomrule
\end{tabular}
\end{table}

This prediction is made \textit{before} any spectroscopic measurement of CDG-2, with no free parameters beyond $a_0 = 1.2 \times 10^{-10}$ m/s$^2$. If follow-up spectroscopy measures $\sigma \approx 17$--$25$ km/s, it would strongly support GCV.

\section{Falsifiable Predictions}
\label{sec:predictions}

\begin{enumerate}
    \item \textbf{Void expansion}: 5--15\% faster than $\Lambda$CDM. Testable with DESI/Euclid by 2028.
    \item \textbf{ISW enhancement}: Factor $\sim1.5$ from supervoids. Testable with improved CMB maps and void catalogs.
    \item \textbf{$w(z)$ shape}: Follows $w(z) = -1 + \epsilon\,\sigma^2(z)\,f_\mathrm{void}(z)$, not linear CPL. Testable with DESI Year-3.
    \item \textbf{CDG-2 velocity dispersion}: $\sigma \approx 21$ km/s (Eq.~\ref{eq:sigma_cdg2}). Testable with spectroscopy of the GC system.
\end{enumerate}

\section{Discussion}
\label{sec:discussion}

\subsection{Summary of Tests}

\begin{table}[h]
\centering
\caption{GCV vs $\Lambda$CDM: summary}
\begin{tabular}{lccc}
\toprule
Test & $\Lambda$CDM & GCV & Status \\
\midrule
SPARC rotation curves (175) & Requires DM halo & 0.06\% deviation & Competitive \\
CDG-2 dark galaxy & Requires DM halo & Deep MOND, $\sigma{\approx}21$ km/s & Prediction \\
CMB acoustic peaks & Excellent & $<0.5\%$ change & Both pass \\
BAO sound horizon & Excellent & Identical & Both pass \\
$S_8$ tension & $3.9\sigma$ & $2.6\sigma$ & GCV better \\
$\Delta\chi^2$ (simplified) & 0 & $-17.70$ & GCV better \\
Full Planck likelihood & Excellent & \textit{Not yet tested} & Unknown \\
Full MCMC & Mature & \textit{Not yet done} & Unknown \\
\bottomrule
\end{tabular}
\end{table}

\subsection{Honest Assessment of Limitations}

\begin{enumerate}
    \item \textbf{Full Planck likelihood}: Not yet computed. May change the $\Delta\chi^2$ result.
    \item \textbf{Full MCMC}: $\mu_0 = 0.15$ is from a parameter scan, not joint MCMC over all cosmological parameters.
    \item \textbf{Derivation of $\mu_0$}: Fitted from data, not derived from the Lagrangian.
    \item \textbf{Non-linear structure}: Halofit/hmcode not computed for GCV.
    \item \textbf{Formal connection}: The bridge between $\chi_v$ (galactic) and $\mu(a)$ (cosmological) is physically motivated but not formally derived.
    \item \textbf{Peer review}: This is a preprint. Results have not been independently verified.
\end{enumerate}

\subsection{Relation to Existing Frameworks}

GCV is related to but distinct from: MOND/TeVeS (GCV provides a physical mechanism and extends to cosmological scales); $f(R)$ gravity (GCV is a specific physical model); the EFT of Dark Energy \citep{Bellini2015} (the $\mu(a)$ parameterization fits within this framework); and emergent gravity \citep{Verlinde2017} (both propose gravity emerging from vacuum effects, with different implementations).

\section{Conclusions}
\label{sec:conclusions}

We present GCV v15.0, a modified gravity framework with the following key results:

\begin{enumerate}
    \item \textbf{Galactic scale}: 175 SPARC galaxies reproduced with 0.06\% mean deviation, no free parameters per galaxy.
    \item \textbf{Cosmological scale}: Modified CLASS yields $\Delta\chi^2 = -17.70$ (preliminary), $S_8$ tension reduced from $3.9\sigma$ to $2.6\sigma$, CMB and BAO preserved.
    \item \textbf{Falsifiable}: Four predictions testable by 2028, including $\sigma \approx 21$ km/s for CDG-2.
\end{enumerate}

GCV is not yet a proven replacement for $\Lambda$CDM. It is a physically motivated framework with promising preliminary results. The most urgent next steps are: full MCMC with official Planck likelihood; derivation of $\mu_0$ from the Lagrangian; non-linear power spectrum; independent verification.

All code is open source at \url{https://github.com/manuzz88/gcv-theory}.

\section*{Acknowledgments}
We thank the SPARC collaboration for publicly available data. This work used CLASS \citep{Blas2011}, Python, NumPy, SciPy, and Matplotlib. We acknowledge AI assistance (Windsurf/Cascade) in code development. Planck 2018 data products \citep{Planck2020} were used for comparison.

\section*{Data Availability}
All code (scripts 119--139) and the modified CLASS C source are at \url{https://github.com/manuzz88/gcv-theory}. SPARC data: \url{http://astroweb.cwru.edu/SPARC/}.

\begin{thebibliography}{99}

\bibitem[Bekenstein(2004)]{Bekenstein2004}
Bekenstein, J.~D. 2004, Phys.\ Rev.\ D, 70, 083509

\bibitem[Bellini \& Sawicki(2015)]{Bellini2015}
Bellini, E., \& Sawicki, I. 2015, JCAP, 07, 050

\bibitem[Blas et al.(2011)]{Blas2011}
Blas, D., Lesgourgues, J., \& Tram, T. 2011, JCAP, 07, 034

\bibitem[DES Collaboration(2022)]{DES2022}
DES Collaboration 2022, Phys.\ Rev.\ D, 105, 023520

\bibitem[Heymans et al.(2021)]{KiDS2021}
Heymans, C., et al. 2021, A\&A, 646, A140

\bibitem[Kov\'acs et al.(2022)]{Kovacs2022}
Kov\'acs, A., et al. 2022, MNRAS, 515, 4417

\bibitem[Lelli et al.(2016)]{Lelli2016}
Lelli, F., McGaugh, S.~S., \& Schombert, J.~M. 2016, AJ, 152, 157

\bibitem[LUX Collaboration(2017)]{LUX2017}
Akerib, D.~S., et al. 2017, Phys.\ Rev.\ Lett., 118, 021303

\bibitem[McGaugh et al.(2016)]{McGaugh2016}
McGaugh, S.~S., Lelli, F., \& Schombert, J.~M. 2016, Phys.\ Rev.\ Lett., 117, 201101

\bibitem[Milgrom(1983)]{Milgrom1983}
Milgrom, M. 1983, ApJ, 270, 365

\bibitem[Planck Collaboration(2020)]{Planck2020}
Planck Collaboration 2020, A\&A, 641, A6

\bibitem[Riess et al.(2022)]{Riess2022}
Riess, A.~G., et al. 2022, ApJ, 934, L7

\bibitem[Verlinde(2017)]{Verlinde2017}
Verlinde, E. 2017, SciPost Phys., 2, 016

\bibitem[Li et al.(2025)]{Li2025}
Li, D., et al. 2025, ApJ Lett., arXiv:2506.15644

\bibitem[Harris et al.(2017)]{Harris2017}
Harris, W.~E., Harris, G.~L., \& Hudson, M.~J. 2017, ApJ, 836, 67

\bibitem[Burkert \& Forbes(2020)]{Burkert2020}
Burkert, A., \& Forbes, D.~A. 2020, AJ, 159, 56

\end{thebibliography}

\appendix

\section{Version History}
\begin{itemize}
    \item v1.0 (2025-11-02): Initial preprint. Galaxy rotation curves, weak lensing, cluster mergers.
    \item v2.1 (2025-11-02): Added MCMC, mass threshold, H0 tension analysis.
    \item v15.0 (2026-02-19): Major update. Lagrangian derivation of $\Gamma(\rho)=\tanh$. Unified formulation. Modified CLASS Boltzmann solver. $\Delta\chi^2=-17.70$. $S_8$ tension reduced $3.9\sigma\to2.6\sigma$.
    \item v15.1 (2026-02-21): Added CDG-2 dark galaxy test case (Section~\ref{sec:cdg2}). Testable prediction: $\sigma \approx 21$ km/s.
\end{itemize}

\section{How to Reproduce}

\begin{verbatim}
# Clone repository
git clone https://github.com/manuzz88/gcv-theory.git
cd gcv-theory

# Run the definitive CLASS test (Script 138)
python3 gcv_gpu_tests/theory/138_CLASS_GCV_Modified_Gravity.py

# Run the CDG-2 dark galaxy test (Script 139)
python3 gcv_gpu_tests/theory/139_CDG2_Dark_Galaxy_GCV_Test.py

# For CLASS C code modification details:
# see gcv_gpu_tests/theory/135_CLASS_Implementation_Blueprint.py
\end{verbatim}

The modified CLASS C code changes are documented in the commit history of the repository.

\end{document}
